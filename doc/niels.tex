%-----------------------------------------------------------------------------
%
%               Template for sigplanconf LaTeX Class
%
% Name:         sigplanconf-template.tex
%
% Purpose:      A template for sigplanconf.cls, which is a LaTeX 2e class
%               file for SIGPLAN conference proceedings.
%
% Guide:        Refer to "Author's Guide to the ACM SIGPLAN Class,"
%               sigplanconf-guide.pdf
%
% Author:       Paul C. Anagnostopoulos
%               Windfall Software
%               978 371-2316
%               paul@windfall.com
%
% Created:      15 February 2005
%
%-----------------------------------------------------------------------------

\documentclass{sigplanconf}

% The following \documentclass options may be useful:

% preprint      Remove this option only once the paper is in final form.
% 10pt          To set in 10-point type instead of 9-point.
% 11pt          To set in 11-point type instead of 9-point.
% authoryear    To obtain author/year citation style instead of numeric.

\usepackage{amsmath}


\begin{document}

\special{papersize=8.5in,11in}
\setlength{\pdfpageheight}{\paperheight}
\setlength{\pdfpagewidth}{\paperwidth}

\conferenceinfo{CONF 'yy}{Month d--d, 20yy, City, ST, Country} 
\copyrightyear{20yy} 
\copyrightdata{978-1-nnnn-nnnn-n/yy/mm} 
\doi{nnnnnnn.nnnnnnn}

% Uncomment one of the following two, if you are not going for the 
% traditional copyright transfer agreement.

%\exclusivelicense                % ACM gets exclusive license to publish, 
                                  % you retain copyright

%\permissiontopublish             % ACM gets nonexclusive license to publish
                                  % (paid open-access papers, 
                                  % short abstracts)

\titlebanner{banner above paper title}        % These are ignored unless
\preprintfooter{short description of paper}   % 'preprint' option specified.

\title{The design and implementation of a numerically intensive expressions language for science}
\subtitle{A pragmatic study}

\authorinfo{Simon A. F. Lund}
           {Niels Bohr Insitute}
           {safl@nbi.ku.dk}

\maketitle

\begin{abstract}
This is the text of the abstract.
\end{abstract}

\category{CR-number}{subcategory}{third-level}

% general terms are not compulsory anymore, 
% you may leave them out
\terms
term1, term2

\keywords
keyword1, keyword2

\section{Introduction}

Experimenting with composition and contraction.

\section{Design}

Primitive types: int[32,64], float[32,64], complex[64,128], bool, and string.
Arrays are first-class cizitens. Should everything be an array by default? Similar to the notion of Matlab?

Array-notation for promoting operators to operations on arrays.
Serial semantics.
No short-hand notations for anything.
Main is a block of statements:
main {
    a = ones as (3,3,3)
    a = 1 as (3,3,3)
    a = zeros as (3,3,3)
    a = 0 (3,3,3)
    a = 1.0 as (3,3,3)
    b = a+1
}

'as' is a shape operator. Taking an array and a shape as argument.
ones as (3,3,3)

Shapes are declared with parenthesis (3,3,3)
Indexing is declared with square brackets []
Ranges are declared with ellipsis 1..1
Slicing is indexing and ranges such as ones[]

Blocks can be nested.

Built-in declarative command-line parsing. Inspirated by Chapel and docopt.

Mathematical ASCII notation, that is, not the extent of APL og unicode.
Inspiration from ISO 80000-2 for set/range/interval notation.

\section{Tools of the trade}

The dragon book
Ambigious grammar?
Context free or context sensitive grammar?
Parser generator or parser combinator?

Parser algorithms LL, LALR(1), LR(1), GLR, lookahead and backtracking.
Capabilities of the tool should match the requirements of the design.

Choice of compiler-compiler, something that will generate C or C++. First off able to construct interpreters easily, secondly compilers. Something worth considering are requirements of the artifact produced by tools. Does it require another interpreter? Virtual machine? Runtime libraries?  Antlr\cite{cc:antlr} seems popular, however, version 4.5 does not support C/C++ as codegen target and various googling/stackoverflow revealed that support was buggy in previous version. So I concluded that the niceities of antlr 4 would not be available, in conclusion not much motivation for using it. Lex\cite{cc:lex}/Yacc\cite{cc:yacc} is a classic, however, unconveniently licences. Flex/Bison since it seems like it gets the job done, decent license, documented, and a wealth of community knowledge.

Lemon\cite{cc:lemon} and parsec\cite{cc:parsec} might be interesting too.  One could of course also choose and implement it by hand, but I just want something well documented and supported. There is most likely better tools out there, something newer but to start out experimenting flex/bison should be fine. BNFC, the backus-naur-form converter, emits 

\section{Performance Study}

Vestibulum vitae consequat sapien. Aliquam malesuada sit amet ipsum ut hendrerit. Nullam sit amet tristique eros. Integer lacus leo, viverra vel velit eu, dapibus aliquam mi. Maecenas commodo non quam a porttitor. Nullam vel sollicitudin tellus. Etiam nec diam nibh. Aenean aliquam arcu non leo lobortis, et tincidunt dolor viverra. Morbi tincidunt magna eu mollis molestie. Duis eu egestas quam.

\section{Future Work}

Curabitur consectetur vel magna in dictum. Aliquam fermentum tortor sed mi sagittis lobortis. Donec eleifend dui sit amet libero pretium posuere. Nam cursus ligula sit amet libero ornare, ac pulvinar diam laoreet. Etiam consectetur magna eu turpis posuere, a mollis arcu consectetur. Aliquam a diam vel metus fringilla tincidunt. Sed euismod vitae mauris sit amet pharetra. Aliquam euismod vehicula orci. Sed volutpat aliquet turpis ut viverra. Curabitur ac rutrum ex. Nunc dapibus ligula tempus mattis vestibulum. Praesent id sapien condimentum, mattis neque a, consectetur nunc. In elementum sed nunc vel feugiat. Sed cursus dictum ante, at tincidunt nibh mollis ac. Cras ultrices nisi dolor, sed ultrices ex dignissim ac.

\section{Conclusion}

Lorem ipsum dolor sit amet, consectetur adipiscing elit. Vestibulum molestie tortor non interdum tempus. Nullam rutrum metus ac varius varius. Integer sit amet ipsum ipsum. Integer vehicula, eros ac laoreet posuere, felis tortor hendrerit nunc, et tincidunt massa tortor ut odio. Curabitur nisi libero, ornare at quam a, vulputate tincidunt lacus. Nulla sodales enim a sapien elementum porttitor. Aenean feugiat gravida libero eu rutrum. Nunc mattis dui eu justo bibendum, nec maximus ligula auctor. Ut orci velit, varius eu lobortis in, scelerisque eget diam. In sapien nunc, rutrum at tincidunt vel, dignissim at tellus. Duis imperdiet sapien dui, eu dignissim ipsum tristique vel. Nunc convallis mattis enim ac pellentesque. Sed ut fermentum libero, at feugiat massa. Cras a mattis nisi. Nullam ultrices enim id luctus interdum.

%\appendix
%\section{Appendix Title}
%
%This is the text of the appendix, if you need one.

\acks

Acknowledgments, if needed.

% We recommend abbrvnat bibliography style.

\bibliographystyle{abbrvnat}
\bibliography{lit}
% The bibliography should be embedded for final submission.
%\begin{thebibliography}{}
%\softraggedright
%
%\bibitem[Smith et~al.(2009)Smith, Jones]{smith02}
%P. Q. Smith, and X. Y. Jones. ...reference text...
%
%\end{thebibliography}

\end{document}

%                       Revision History
%                       -------- -------
%  Date         Person  Ver.    Change
%  ----         ------  ----    ------

%  2013.06.29   TU      0.1--4  comments on permission/copyright notices


